% http://www.ctan.org/tex-archive/macros/latex/contrib/beamer/examples
% http://latex.artikel-namsu.de/english/beamer-examples.html

%\documentclass{beamer}
\documentclass[usenames,dvipsnames]{beamer}
\usepackage{amsmath}
\usepackage{amssymb}
\usepackage{bm}
\usepackage{fancybox, graphicx}
\usepackage{listings}
\usepackage{tikz} % Diagrams
\usepackage{color}
\usepackage{textcomp} % See https://tex.stackexchange.com/questions/145416/how-to-have-straight-single-quotes-in-lstlistings

\lstset{language=bash,upquote=true} % Format listings as appropriate for bash. Inexplicably we get problems if the language is set as part of the \begin{lstlisting} command.

% https://tex.stackexchange.com/questions/36030/how-to-make-a-single-word-look-as-some-code
\definecolor{light-gray}{gray}{0.95}
\newcommand{\code}[1]{\colorbox{light-gray}{\texttt{#1}}}



\usetheme{boxes}
\usecolortheme{beaver}


\title{Introduction to Computing with GPUs}
\author{Lorne Whiteway \\ lorne.whiteway@star.ucl.ac.uk}
\institute{Astrophysics Group \\ Department of Physics and Astronomy \\ University College London}
\date{August 2018}

\subject{IT}

\begin{document}

\frame{\titlepage}


\begin{frame}{Where to find this presentation}
  \begin{block}{}
    Find the presentation at \alert{\url{https://tinyurl.com/XXXXXXX}}.\\
  \end{block}
  \begin{block}{}
    On this page click on `Download' to get a copy of the presentation.
  \end{block}
\end{frame}

\begin{frame}{Motivation}
  \begin{block}{}
    This presentation summarises material from the (excellent) course `CUDA Programming on NVIDIA GPUs' by Mike Giles, Oxford.
  \end{block}
  \begin{block}{}
    \url{http://people.maths.ox.ac.uk/~gilesm/cuda/}
  \end{block}
\end{frame}

\begin{frame}{Motivation}
  \begin{block}{}
    \begin{itemize}
      \item{This presentation summarises material from the (excellent) course `CUDA Programming on NVIDIA GPUs' by Mike Giles, Oxford.}\\~\
      \item{\url{http://people.maths.ox.ac.uk/~gilesm/cuda/}}
    \end{itemize}
  \end{block}
\end{frame}

\begin{frame}{Two paradigms}
  \begin{block}{}
    \begin{itemize}
      \item{Intel (market cap \$220B\footnotemark): One-chip-fits-all. Use the same chip (a CPU) for all tasks (e.g. both word processing and high-performance computing (HPC)).}\\~\
      \item{NVIDIA (market cap \$152B\footnotemark[\value{footnote}]): Create a specialised chip (the GPU) specifically tailored for certain tasks (including HPC).}
    \end{itemize}
  \end{block}
  \footnotetext[\value{footnote}]{As of 27 July 2018}
\end{frame}

\begin{frame}{GPUs}
  \begin{block}{}
    \begin{itemize}
      \item{GPU  = graphics processing unit}
      \item{Specialised chip}
      \item{Has many processors and a specialised memory structure}
      \item{Designed specifically for high performance via parallelisation when doing graphics rendering.}
      \item{Single instruction, multiple data (SIMD).}
      \item{This design also makes them suitable for parallelizable problems in HPC.}
    \end{itemize}
  \end{block}
\end{frame}

\begin{frame}{GPUs are displacing CPUs for HPC}
  \begin{block}{}
    \begin{itemize}
      \item{GPUs offer better value (FLOPS/\$) and energy usage (FLOPS/J) than CPUs.}\\~\
      \item{5 of the world's top 7 supercomputers use GPUs.\footnotemark}\\~\
      \item{All else being equal, we should probably use GPUs for HPC...}
    \end{itemize}
  \end{block}
\footnotetext[\value{footnote}]{As of June 2018}
\end{frame}

\begin{frame}{But...}
  \begin{block}{}
    \begin{itemize}
      \item{Installed base of CPUs (e.g. splinter).}\\~\
      \item{Takes time to develop GPU expertise.}\\~\
      \item{Program code needs to be altered to run (effectively) on GPUs.}
    \end{itemize}
  \end{block}
\end{frame}

\begin{frame}{GPUs on splinter}
  \begin{block}{}
    \begin{itemize}
      \item{New (July 2018) GPU `Tesla V100' (in addition to old `K80').}\\~\
      \item{Thanks to Ofer (funding) and Edd (implementation).}\\~\
      \item{How can we use these cards effectively?}
    \end{itemize}
  \end{block}
\end{frame}


\end{document}
